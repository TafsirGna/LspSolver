\documentclass[10pt,a4paper]{article}
\usepackage[utf8]{inputenc}
\usepackage{amsmath}
\usepackage{amsfonts}
\usepackage[french]{babel}
\usepackage{amssymb}
\usepackage{graphicx}
\usepackage{sectsty}
\usepackage[colorlinks=true,linkcolor=black]{hyperref}
\usepackage{color}
\pagestyle{headings}
\author{Tafsir GNA}

%\allsectionsfont{\sf}
%\sectionfont{\sffamily\color{black}\sectionrule{3ex}{3pt} %
%{-1.5ex}{1pt}}

\begin{document}

\setcounter{section}{0}

	\begin{titlepage}
		\centering
		
		\noindent%
		
		\begin{minipage}{.15\linewidth}
			\includegraphics[scale=0.4]{img/ifri_logo.png}
		\end{minipage}
		\hfill
		\begin{minipage}{.68\linewidth}\centering
			\textsc{République du Bénin}\\
			\vspace*{.5cm}
			\textsc{\Large Université d'Abomey-Calavi}
		\end{minipage}
		\hfill
		\begin{minipage}{.15\linewidth}
			\includegraphics[width=1\linewidth]{img/uac_logo.png}
		\end{minipage}
				
		\vspace{1cm}{\scshape\Large Institut de Formation de Recherche en Informatique (IFRI)\par}
		\vspace{1cm}{\scshape\Large Mémoire pour l'obtention du diplôme de Master en Systèmes d'Information et Réseaux Informatiques\par}
		\vspace{1.5cm}{\huge\bfseries Résolution de "Pigment Sequencing Problem" avec les algorithmes génétiques\par}
		\vspace{2cm}{\scshape \Large\textbf{Présenté par} \\ Tafsir GNA \par}
		\vfill
		\Large\textbf{Supervisé par}\par Dr. Norbert \textsc{Hounkonnou} \\ \& \\ Ir. Ratheil Vinasétan \textsc{Houndji}, PhD Student
		\vfill
		{\large \scshape Année Académique 2016-2017 \par}
	\end{titlepage}

	% Tables des matières
	\tableofcontents
	
	\newpage
	
	%Liste des figures
	\listoffigures
	
	\newpage
	
	%Liste des Tableaux
	\listoftables
	
	\newpage
	
	\part*{Introduction}
	\addcontentsline{toc}{part}{Introduction}
	
	\newpage
	
	\part{Etat de l'art}
		\section*{Introduction}
		\addcontentsline{toc}{section}{Introduction}
		\section*{Conclusion}
		\addcontentsline{toc}{section}{Conclusion}
		
	\newpage
	
	\part{Modèles, Formulations et Implémentation}
		\section*{Introduction}
		\section*{Conclusion}
		
	\newpage
	
	\part{Expérimentations et Analyse des résultats}
		\section*{Introduction}
		\section*{Conclusion}
		
	\newpage
		
	\part*{Conclusion}
	\addcontentsline{toc}{part}{Conclusion}
	
	\part*{Bibliographie}
\end{document}