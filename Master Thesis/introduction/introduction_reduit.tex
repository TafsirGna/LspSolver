Dans un processus de planification de production, le problème de dimensionnement
de lots (lot sizing) consiste à identifier les articles à produire, quand il
faut les produire et sur quelle machine de façon à satisfaire les demandes
tout en considérant les objectifs financiers. Connu dans la
littérature sous le nom de problème de lot sizing, il a été beaucoup étudié
ces dernières décennies \cite{cathy}.\\
	\hspace*{.5cm} Différentes versions de dimensionnement de lots ont été proposées dans la littérature, chacune étant spécifique à leur domaine d'application. Récemment, Houndji et al. \cite{hvr_stockingCost} et Ceschia et al. \cite{opthub} ont expérimenté une variante NP-Difficile du problème de dimensionnement de lots. Cette version est connue sous le nom de \emph{Pigment Sequencing Problem} (PSP) (Pochet et Wolsey \cite{pochet_wolsey}) et a été récemment inclue à la bibliothèque CSPlib (Gent and Walsh, \cite{gent_walsh}) . Il s'agit de produire plusieurs articles avec une seule machine dont la capacité de production est limitée à un article par période. L'horizon de planification est discret et fini, et il y a des coûts de stockage et des coûts de transition d'une production à une autre. Par ailleurs, les demandes sont normalisées et donc binaires. \\
	\hspace*{.5cm} Toutefois, la résolution d'un PSP se heurte comme tout problème de dimensionnement de lots à des difficultés. Ainsi, une ressource de production n'est le plus souvent pas seulement dédiée à un unique article, mais plutôt utilisée pour produire différents types d'articles. Aussi, un plan de production doit remplir plusieurs objectifs parfois contradictoires, notamment, garantir un excellent niveau du service-client et minimiser la production et les coûts de stockage. Dans ce contexte, la quête de méthodes de recherche toujours plus efficaces a guidé les différentes recherches effectuées\cite{ratheil_master} \cite{ceschia} dans le domaine. \\
	\hspace*{.5cm} Un problème d’optimisation tel que le problème de dimensionnement de lots se divise en deux phases : recherche des solutions admissibles puis recherche de la solution à coût optimal parmi ces dernières. L’usage d’un algorithme génétique est adapté à une exploration rapide et globale d’un espace de recherche de taille importante et est capable de fournir plusieurs solutions. Ainsi, l'utilisation des algorithmes génétiques pour des problèmes de dimensionnement de lots semble raisonnable dans le cas de grands problèmes où trouver la solution avec d'autres algorithmes reste encore problématique en temps.   
	
	\section*{Contribution}
	Le \emph{Pigment Sequencing Problem} (PSP) est un problème NP-Difficile d'optimisation combinatoire pour lequel les instances de taille moyenne peuvent être efficacement résolues en utilisant une formulation appropriée de programmation en nombre mixte \cite{pochet_wolsey}. Cependant, dans notre revue de littérature, aucun modèle basé sur les algorithmes génétiques n'a encore été proposé pour ce problème. Nous proposons donc deux méthodes de recherche basées sur les algorithmes génétiques pour le problème. La première est une méthode appelée \emph{Hierarchical Coarsed-grained and Master-slave Parallel Genetic Algorithm} (HCM-PGA) qui divise la population globale en sous-populations et qui confie l'évaluation des différents chromosomes à des processus fils. La seconde, appelée \emph{Hierarchical Fine-grained and Coarse-grained Parallel Genetic Algorithm} (HFC-PGA) qui réduit le champ de croisement à l'environnement immédiat tout en échangeant des individus avec les processus voisins. Les tests que nous avons menés afin de valider nos deux méthodes montrent que les algorithmes génétiques pourraient être efficaces pour résoudre le PSP. 
	 
	\section*{Organisation du travail}

	Le travail effectué est organisé dans ce document en 3 chapitres. Le premier chapitre présente une revue de littérature des problèmes de dimensionnement de lots en planification de production en nous concentrant sur les classes de problèmes qui nous concernent dans notre étude. Ensuite, nous présentons le PSP, le décrivons et présentons les modèles et méthodes utilisés dans sa résolution. Nous présentons également dans ce chapitre les algorithmes génétiques, leurs étapes ainsi que leur fonctionnement. Dans le deuxième chapitre, nous détaillons les deux méthodes de recherche utilisées ainsi que les algorithmes associés utilisés. Ensuite, dans le dernier et troisième chapitre, nous expérimentons  nos deux méthodes, présentons les résultats obtenus et comparons ces résultats à l'état de l'art en la matière. Pour finir, nous tirons une conclusion du travail effectué et dressons les perspectives possibles en vue d'améliorer ce qui a été fait.