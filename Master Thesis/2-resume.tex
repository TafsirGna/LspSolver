\resume
\begin{abstract}

Le dimensionnement de lots est la tâche la plus difficile et la plus importante dans les problèmes de planification de production en industrie. Il consiste à identifier, sur un horizon de planification, le nombre (et le type) d’articles à produire et à quel moment produire de façon à minimiser le coût total de production. De récentes recherches ont expérimenté une variante NP-Difficile du problème de dimensionnement de lots : le “Pigment Sequencing Problem” (PSP). Aussi, les algorithmes génétiques ont montré leur efficacité sur des problèmes d’optimisation difficiles en trouvant de relativement bonnes solutions par rapport aux solutions optimales. Ce travail vise donc à travers l'application de deux méthodes de recherche basées sur les algorithmes génétiques au “Pigment Sequencing Problem”, à identifier un plan de production qui respecte les capacités de production des machines tout en minimisant les coûts de stockage et de transition d’une production à une autre.

\textbf{Mots clés}:
\emph{Algorithme génétique}, \emph{planification de production}, \emph{pigment sequencing problem}, \emph{dimensionnement de lots}
\end{abstract}

\selectlanguage{english}
\begin{abstract}

The lot sizing is the most difficult and important task in production planning problems in industry field. It involves determining through a planning horizon the number (and the type) of the items to manufacture and when to manufacture these items in order to minimize the overall cost of manufacturing. Some recent researches have experienced an NP-hard variant of lot sizing problem called "Pigment Sequencing Problem" (PSP). Besides, Genetic Algorithms (GA) have shown how effecient they are, in finding some relatively good solutions in comparison to the optimal ones. Thus, This work aims to apply two heuristic approaches based on genetic algorithms to the Pigment Sequencing Problem, to find a solution that fits into the machine capacity restrictions and that minimizes the stocking costs and change-over costs from one item to another one.

\textbf{Key words}: 
\emph{Genetic algorithm}, \emph{production planning}, \emph{pigment sequencing problem}, \emph{lot sizing}

\end{abstract}
