\begin{abstract}
Le travail présenté dans ce document vise à appliquer deux approches de résolution de problèmes basées sur les algorithmes génétiques à un problème typique de dimensionnement de lots : le "Pigment Sequencing Problem" (PSP). A cette fin, nous commençons par donner un aperçu général des problèmes de dimensionnement de lots en planification de production. Nous continuons en introduisant le PSP. Nous présentons les différents modèles et méthodes de résolution ayant été appliqués à ce problème au nombre desquelles figurent l'approche basée sur la programmation par contrainte et celle basée sur le recuit simulé. Le document se poursuit en faisant un état de l'art générique de la méthode de résolution de problèmes que sont les algorithmes génétiques. Nous avons ainsi pu appliquer une catégorie d'algorithmes génétiques dits algorithmes génétiques parallèles  et hiérarchiques. Les tests effectués nous ont permis de comparer cette dernière méthode de résolution à celles déjà appliqués dans de précédentes recherches.

\textbf{Mots clés}: \emph{Algorithme génétique, planification de production, pigment sequencing problem, dimensionnement de lots}.
\end{abstract}