\begin{abstract}

Le dimensionnement de lots tient une place importante en planification de production en industrie. Il consiste à trouver un plan de production qui à la fois satisfait les demandes et prend en compte les objectifs financiers de l'entreprise. De récentes recherches ont testé une variante NP-Difficile du problème de dimensionnement de lots : Le \emph{Pigment Sequencing Problem} (PSP). Différentes méthodes de résolution de problèmes ont ainsi été appliquées au PSP. Aucune des méthodes appliquées n'est basée sur les algorithmes génétiques qui ont pourtant montré leur efficacité sur nombre de problèmes d'optimisation. Dans ce travail, nous appliquons deux méthodes de résolution de problèmes basées sur les algorithmes génétiques au PSP. Ainsi, nous donnons un aperçu général des problèmes de dimensionnement de lots en planification de production et présentons le PSP ainsi que les différents modèles et méthodes de résolution ayant été appliqués à ce problème. Le document fait un bref état de l'art de la méthode de résolution de problèmes que sont les algorithmes génétiques. Nous avons appliqué une catégorie d'algorithmes génétiques dits algorithmes génétiques parallèles et hiérarchiques. Les tests effectués nous ont permis de comparer cette dernière méthode de résolution à celles déjà appliquées dans de précédentes recherches.

\textbf{Mots clés}: \emph{Algorithme génétique, planification de production, pigment sequencing problem, dimensionnement de lots}.
\end{abstract}