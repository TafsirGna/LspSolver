\begin{abstract}

Lot sizing takes an important place in production planning in industry. It consists in determining a production plan that at the same time meets the orders and takes into account the financial objectives of the enterprise. Recent researches have experienced a NP-Hard variant of lot sizing problem: the \emph{Pigment Sequencing Problem} (PSP). Several methods have been tested on PSP. None of the tested methods is based on genetic algorithms whereas they showed their efficiency in solving optimization problems. In this document, we apply two solving methods based on genetic algorithms on PSP. Doing that, we give a short view of the lot sizing problems in production planning, introduce PSP and the different methods and models applied yet to this particular problem. This document also does a short literature review of the solving method that are genetic algorithms. We applied a category of genetic algorithms which is the hierarchical and parallel genetic algorithms. The trials performed allow us to compare this last solving method to the ones applied in ealier researches.  

\textbf{Keys words}: \emph{Genetic algorithm, production planning, pigment sequencing problem, lot sizing}.
\end{abstract}