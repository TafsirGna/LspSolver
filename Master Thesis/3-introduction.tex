\section*{Introduction}
\hspace*{.5cm} Dans un processus de planification de production, le dimensionnement
de lots (lot sizing) qui consiste à identifier les articles à produire, quand il
faut les produire et sur quelle machine de façon à satisfaire les demandes
tout en considérant les objectifs financiers, est l’activité la plus importante
et la plus difficile. En effet, de la qualité des décisions qui seront prises,
dépendent la performance et la compétitivité de l’entreprise. Connu dans la
littérature sous le nom de problème de lot sizing, il a été beaucoup étudié
ces dernières décennies depuis les travaux de Wagner et Within en 1958. \\
	\hspace*{.5cm} Différentes versions de dimensionnement de lots ont été proposées dans le littérature, chacune étant spécifique à leur domaine d'application. Récemment, Houndji et al. \cite{hvr_stockingCost} et Ceschia et al. \cite{opthub} ont expérimenté une variante NP-Difficile du problème de dimensionnement de lots. Cette version est connue comme le “Pigment Sequencing Problem” (Pochet et Wolsey \cite{pochet_wolsey}) et a été récemment inclue à la bibliothèque CSPlib (Gent and Walsh, \cite{gent_walsh}) . Il s'agit de produire plusieurs articles avec une seule machine dont la capacité de production est limitée à un article par période. L'horizon de planification est discret et fini, et il y a des coûts de stockage et des coûts de transition d'une production à une autre. Par ailleurs, les demandes sont normalisées et donc binaires. \\
	\hspace*{.5cm} Dans la plupart des cas, un problème d’optimisation tel que le problème de dimensionnement de lots se divise naturellement en deux phases : recherche des solutions admissibles puis recherche de la solution à coût optimal parmi ces dernières. Suivant la méthode employée, ce découpage est plus ou moins apparent dans la résolution. L’usage d’un algorithme génétique est adapté à une exploration rapide et globale d’un espace de recherche de taille importante et est capable de fournir plusieurs solutions. Des calculs effectués ont révélé que l'utilisation des algorithmes génétiques pour des problèmes de dimensionnement de lots semble raisonnable dans le cas de grands problè-mes où trouver la solution avec d'autres algorithmes reste encore problématique en temps. Ainsi, Il est possible qu'un algorithme génétique convenablement construit et paramétré puisse trouver de bien meilleurs résultats comparativement à ceux trouvés par d'autres algorithmes tels que le programmation en nombre entier et cela pour un même laps de temps considéré.
	
	\section*{Notre contribution}
	\hspace*{.5cm} Le "Pigment Sequencing Problem" est un problème NP-Difficile d'optimisation combinatoire pour lequel les instances de taille moyenne peuvent être efficacement résolues en utilisant une formulation appropriée de programmation en nombre mixte. Cependant, dans notre revue de littérature, aucun modèle basé sur les algorithmes génétiques n'a encore été proposé pour ce problème. Nous proposons donc deux métho-des basées sur les algorithmes génétiques pour le problème. La première est une méthode appelée "Hierarchical Coarsed-grained and Master-slave Parallel Genetic Algorithm" (HCM-PGA) qui divise la population globale en sous-populations et qui confie l'évalua-tion des différents chromosomes à des processus fils. La seconde, appelée "Hierarchical Fine-grained and Coarse-grained Parallel Genetic Algorithm" (HFC-PGA) qui réduit le champ de croisement à l'environnement immédiat tout en échangeant des individus avec les processus voisins. Les tests que nous avons menés afin de valider nos deux méthodes ont produit des résultats satisfaisants et confirmé le fait que les algorithmes génétiques sont efficaces quant il s'agit de résoudre des problèmes d'optimisation. 
	 
	\section*{Organisation du travail}

	\hspace*{.5cm} Le travail effectué est organisé dans ce document en 3 parties. Le première partie présente une revue de littérature des problèmes de dimensionnement de lots en planification de production en nous concentrant sur les classes de problèmes qui nous concernent dans notre étude. Ensuite, nous présentons le PSP, le décrivons et présentons les modèles utilisés dans sa résolution. Nous présentons également dans cette partie les algorithmes génétiques, leurs étapes ainsi que leur fonctionnement. En deuxième partie, nous détaillons les deux méthodes de recherche utilisées ainsi que les algorithmes associés utilisés. Ensuite, en dernière et troisième partie, nous expérimentons  nos deux méthodes, présentons les résultats obtenus et comparons ces résultats à l'état de l'art en la matière. Pour finir, nous tirons une conclusion du travail effectué et dressons les perspectives possibles en vue d'améliorer ce qui a été fait.